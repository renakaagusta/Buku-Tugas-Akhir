\chapter{PENUTUP}
\label{chap:penutup}

\section{Kesimpulan}
\label{sec:kesimpulan}

Berdasarkan pengembangan dan pengujian yang telah dilakukan dari NFT Marketplace NFT untuk Metaverse berbasis Blockchain Ethereum, diperoleh beberapa kesimpulan sebagai berikut:
1. Pengguna NFT Marketplace dapat melakukan pembelian, penyewaan, hingga pembagian kepemilikan pada \emph{NFT} yang dimiliki sehingga meningkatkan kemanfaatan dan keuntungan atas \emph{NFT} yang mereka miliki. 
2.  Hasil pengujian gas menunjukan fungsi-fungsi tambahan yang dikembangkan seperti penyewaan dan pembagian kepemilikan sudah cukup efisien jika dibandingkan fungsi bawaan yang disediakan oleh ERC-1155.
3. Selain itu sistem juga dapat diintegrasikan dengan \emph{platform} lain melalui \emph{API} yang disediakan dengan baik.

\section{Saran}
\label{chap:saran}

\begin{enumerate}[nolistsep]
\item \emph{API} yang dikembangkan saat ini masih tidak terproteksi sehingga apabila penggunaannya terlalu masif maka akan berpotensi menurunkan performa \emph{server}. Hal ini dapat diatas dengan adanya \emph{access key} yang diperlukan sebelum \emph{platform} lain menggunakan \emph{API} yang disediakan
\item Pengembangan \emph{NFT Marketplace} untuk \emph{platform mobile} mengingat mayoritas pengguna internet saat ini mengakses internet melalui \emph{device} \emph{mobile} seperti \emph{smartphone}
\item Integrasi dengan sistem orkestrasi seperti \emph{Kubernetes} untuk menunjang skalibilitas dari aplikasi
\item Pembuatan \emph{automation testing} untuk meminimalisir adanya \emph{bug} apabila terdapat pengembangan lebih lanjut
\end{enumerate}