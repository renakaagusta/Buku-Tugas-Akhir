\begin{center}
  \large
  \textbf{\emph{MARKETPLACE NFT} UNTUK \emph{METAVERSE} BERBASIS \emph{SMART CONTRACT} \emph{BLOCKCHAIN ETHEREUM}}
\end{center}
\addcontentsline{toc}{chapter}{ABSTRAK}
% Menyembunyikan nomor halaman
\thispagestyle{empty}

\begin{flushleft}
  \setlength{\tabcolsep}{0pt}
  \bfseries
  \begin{tabular}{ll@{\hspace{6pt}}l}
  Nama Mahasiswa / NRP&:& Renaka Agusta / 07211940000011\\
  Departemen&:& Teknik Komputer FTEIC - ITS\\
  Dosen Pembimbing&:& 1. Mochamad Hariadi, S.T., M.Sc., Ph.D.\\
  & & 2. Dr. Surya Sumpeno, S.T, M.Sc.\\
  \end{tabular}
  \vspace{4ex}
\end{flushleft}
\textbf{Abstrak}

% Isi Abstrak
\emph{Non Fungible Token} (NFT) adalah jenis token dalam \emph{blockchain} yang merepresentasikan kepemilikan dari suatu aset digital. 
Pada awal kemunculannya NFT sering digunakan untuk menjamin keorisinalitasan suatu karya seni digital. 
Akan tetapi, penggunaan NFT sudah mulai meluas menjadi berbagai hal seperti contohnya adalah aset dalam game yang mengusung konsep \emph{metaverse}. 
\emph{Metaverse} sendiri secadara sederhana merupakan suatu konsep yang menghubungkan realitas dan dengan dunia digital. 
Implementasi NFT dan \emph{blockchain} pada \emph{metaverse} didasarkan pada keuntungan yang diperoleh seperti contohnya adalah adanya mata uang yang bersifat universal dengan \emph{cryptocurrency}, keamanan sistem, desentralisasi, orisinalitas suatu aset dan sebagainya. 
Meskipun banyak keuntungan yang diperoleh, terdapat berbagai hal yang menjadi kekurangan dimana fitur-fitur \emph{NFT Marketplace} yang ada saat ini masih sangat terbatas dengan implmentasi fungsi-fungsi dasar yang disediakan oleh \emph{smart contract} ERC-721 sehingga menyebabkan potensi pemanfaatan \emph{NFT} atas suatu kepemilikan aset juga terbatas seperti pemilik hanya dapat memperoleh keuntungan melalui penjualan dimana hal ini berbeda dengan kepemilikan atas aset di dunia nyata dimana pemilik dapat menyewakan aset yang dimiliki atau bahkan membagi kepemilikannya dengan orang lain. Tentunya hal ini bertentangan dengan konsep \emph{metaverse} yang berupaya untuk membawa dunia nyata ke dunia digital. Selain itu hal ini juga menyebabkan likuiditas dari \emph{NFT} rendah dikarenakan hanya dapat diperjual belikan dan dimiliki kalangan tertentu. Likuiditas menjadi hal yang penting dikarenakan likuiditas merupakan salah satu faktor pembentukan harga aset tak terkecuali \emph{NFT} sebagai aset digital dalam \emph{metaverse}.  
Dikarenakan hal tersebut dibutuhkan suatu \emph{NFT Marketplace} yang menghadirkan berbagai fitur yang mampu meningkatkan potensi keuntungan atas kepemilikan aset dan likuiditas layaknya di dunia nyata yang sejalan dengan konsep \emph{metaverse} seperti sistem sewa dan kepemilikan bersama.

\vspace{2ex}
\noindent
\textbf{Kata Kunci: \emph{Blockchain, Ethereum, NFT, NFT Marketplace, Smart Contract, Metaverse}}